\documentclass[a4paper,10pt]{article} % na razie jako article, jak będzie więcej treści zmieni się na report.

\usepackage[polish]{babel}
\usepackage[utf8]{inputenc}
\usepackage{polski}
\usepackage[T1]{fontenc}

\usepackage{indentfirst}
\usepackage{amsmath, amstext} % ,amsfonts,amssymb
\usepackage{verbatim}
\usepackage{booktabs}
\usepackage{color}
\usepackage[usenames,dvipsnames,svgnames]{xcolor}
\usepackage{utopia}
\usepackage{geometry}
\geometry{verbose,lmargin=3cm,rmargin=3cm}
\frenchspacing

% podtytuł
\usepackage{titling}
\newcommand{\subtitle}[1]{
 \posttitle{
  \par\end{center}
  \begin{center}\large#1\end{center}
  \vskip0.5em}
}

%opening
\title{Nowe Trendy w Obliczeniach Neuronowych}
\subtitle{Spike and Slab Restricted Boltzmann Machine w klasyfikacji obiektów ze zbioru CIFAR-10}
\author{Konrad Brus \\ Jakub A. Gramsz}

\begin{document}
\maketitle

\begin{abstract}
Projekt ma na celu implementację Gaussian Restricted Boltzmann Machine oraz Spike and Slab RBM (zaprezentowanych w \cite{courville2013spike}) i wykorzystanie ich do klasyfikacji obiektów przedstawionych na obrazach. Implementacja uczona oraz testowana będzie na zbiorze danych CIFAR-10 (przedstawiony w \cite{cifar}) przy wykorzystaniu algorytmu uczenia Contrastive Divergence.
\end{abstract}

\section{Wstęp teoretyczny}

\section{Opis problemu}

\section{Opis modelu}

\section{Opis algorytmu uczenia}

\section{Opis eksperymentu (zbioru danych)}

\section{Dostrajanie modelu i wyniki}

\section{Dyskusja i wnioski}


\bibliographystyle{abbrv}
\nocite{*} % na razie póki mało co jest cytowane (powoduje że wszystko z listy artykułów jest tu widoczne
\bibliography{lit}

\end{document}
